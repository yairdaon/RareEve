\documentclass[11pt]{amsart}
\usepackage{amsfonts,amsbsy}
\usepackage[pdftex]{graphicx}
\usepackage{amsaddr}
\usepackage{setspace}
\usepackage{caption}
%\usepackage[margin=1.5in]{geometry}

\usepackage[framemethod=tikz,innerleftmargin=0]{mdframed}




\newcommand{\N}{\mathcal{N}}
\newcommand{\want}{p^{\epsilon}}
\newcommand{\noise}{\sqrt{\epsilon}}




\newcommand{\kknote}[1]{{\textcolor{blue}{#1}}}
\newcommand{\ydnote}[1]{{\textcolor{cyan}{#1}}}
\newcommand{\dmnote}[1]{{\textcolor{green}{#1}}}

% bibliography
% \usepackage[style=authoryear,backend=bibtex,natbib=true]{biblatex}

% \renewbibmacro{in:}{}

% \addbibresource{refs.bib}

% theorem commands 


\begin{document}

\title{Exercise}

\author{Yair Daon}
\address{Courant Institute of Mathematical Sciences \\ New York University \\ 251 Mercer St., New York, NY}


\thanks{Me!!!} 
\date{\today}

\begin{abstract}
  A write up of the exercise for Eric's class.
\end{abstract}


\maketitle


\section{Hi}
\ydnote{This is my note}. \kknote{This is KKK's note}. \dmnote{Guess what this is}.
%\onehalfspacing
\section{Problem}
Suppose you have a particle moving according to the following SDE:

\begin{align}
  dX_t = \noise dW_t,
\end{align}
with $W_t$ Brownian motion. The process starts at $x \in (0,1)$. We are
interested in 

\begin{align}
  \want = \Pr( |X_1| > 1 ) =:\Pr(A). 
\end{align}

\section{Naiive estimator}
The naive estimator will require an exponential number of samples since 
$X_1 \sim \N( x, \epsilon )$ and $A$ is a rare event.

\section{Second Estimator}
We can use the instanton 
\begin{align}
  \phi(t) = (1-t)x + t
\end{align}
and generate the biased process
\begin{align}
dY_t = \dot{\phi}(t)dt + \noise dW_t
\end{align}
which also starts at $x$. The process is
\begin{align}
  Y_t = (1-t)x + t + \noise W_t
\end{align}
and
\begin{align}
  Y_1 = 1 + \noise W_1.
\end{align}
Girsano'v theorem states

Using girsanov's theorem, we find the change of measure factor to be
\begin{align}

\end{align}

%\printbibliography

\end{document}

